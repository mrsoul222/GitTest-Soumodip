\documentclass[12pt]{article}
\usepackage[utf8]{inputenc}
\usepackage{fancyhdr}
\usepackage{geometry}
\usepackage{newunicodechar}
\newunicodechar{₂}{$_2$}
\geometry{margin=1in}

\pagestyle{fancy}
\fancyhf{}
\fancyhead[C]{The Rising Heat: A Global Warning We Can’t Ignore}
\fancyfoot[C]{Page \thepage}

\title{\textbf{The Rising Heat: A Global Warning We Can’t Ignore} \\[1ex]
\large Maulana Abul Kalam Azad University}
\author{Soumodip Some}
\date{25-06-2025}

\begin{document}

\maketitle
\centering
\section{\centering Understanding the Causes of Global Warming}

Global warming refers to the long-term rise in Earth’s average surface temperature due to human activities. One of the major causes is the increased emission of greenhouse gases like carbon dioxide (CO₂), methane, and nitrous oxide, which trap heat in the atmosphere. These gases are mostly released from burning fossil fuels such as coal, oil, and gas, used in vehicles, industries, and electricity production. Deforestation also contributes greatly to global warming, as trees play a vital role in absorbing CO₂. When forests are cleared for agriculture or urban development, not only is this natural carbon sink removed, but the carbon stored in the trees is released back into the air. Other contributors include overuse of aerosols, poor waste management, and industrial emissions. As a result, the planet is getting warmer at an unnatural pace, disrupting climate patterns and threatening ecosystems. Understanding these causes is essential if we want to create meaningful change. Only by recognizing our role in this crisis can we begin to reverse the damage already done and protect the future of our planet.
\begin{center}
    

\section{\centering Impact and the Urgent Need for Action}

The impact of global warming is already visible and growing rapidly. Rising global temperatures are leading to the melting of polar ice caps, causing sea levels to rise and threatening low-lying coastal areas with floods. Extreme weather events, such as intense heatwaves, droughts, hurricanes, and wildfires, are becoming more frequent and severe. These changes not only harm the environment but also affect human lives—damaging crops, increasing the spread of diseases, and displacing entire communities. Animal and plant species are also struggling to adapt, leading to a loss of biodiversity. The good news is, it’s not too late. Actions like using renewable energy, reducing plastic and fossil fuel use, planting more trees, and raising awareness can make a real difference. Governments, businesses, and individuals must work together for climate solutions. Everyone has a role to play—from switching off unused lights to supporting eco-friendly policies. The urgency is clear: if we don’t act now, the consequences will be devastating and irreversible. The time to act on global warming is not tomorrow—it’s today.
\end{center}
\end{document}
